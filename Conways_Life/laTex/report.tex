\documentclass{article}
\usepackage[accepted]{icml2025} % Template ICML 2025
\usepackage[utf8]{inputenc}
\usepackage{graphicx}
\usepackage{amsmath}
\usepackage{booktabs}
\usepackage{hyperref}
\usepackage{caption}
\usepackage{float}
\usepackage{amssymb}



\icmltitlerunning{Conway's Game of Life (GoL)}

\begin{document}

\twocolumn[
\icmltitle{Conway's Game of Life (GoL)}

\begin{icmlauthorlist}
\icmlauthor{Santiago Álvarez Geanta}{ua}
\end{icmlauthorlist}

\icmlaffiliation{ua}{Universidad de Alicante, Ingeniería en Inteligencia Artificial, Grupo 2}
\icmlcorrespondingauthor{Santiago Álvarez Geanta}{sag82@alu.ua.es}
\icmlkeywords{Machine Learning}
\vskip 0.3in
]

\begin{abstract}
In this experiment, we will test GoL in different types of Machine Learning (ML) applications.
Specifically, we will use GoL for embedding and clustering tasks.
\end{abstract}

\section{GoL as an embedding generator}
\subsection{Pipeline}
In our experimental approach, we first binarize the MNIST images and use them as initial states 
on a grid where we apply the classical rules of Conway's Game of Life (GoL). After a fixed number 
of time steps, we take the final state of the cellular automaton as input to a CNN trained for 
classification.

Formally, the considered pipeline can be expressed as:
\begin{center}
MNIST $\rightarrow$ Conway $\rightarrow$ CNN
\end{center}


\subsection{Results}
To achieve this the objective, we use the MNIST dataset \cite{mnist}. Additionally, for the classification task,
we employ a Convolutional Neural Network (CNN).

A standard CNN trained directly on the original MNIST images of size $28 \times 28$ consistently
achieves accuracies in the range of $97\%$ to $99\%$.

When training the model on images transformed by Conway’s Game of Life, we observe a significant drop in accuracy,
falling from values close to $97\%$ to approximately $35\%$. While this result is clearly lower than the baseline,
it is still above random chance (which would be $10\%$ in a 10-class problem).

\subsection{Analysis}
As a consequence, the structure that defines a digit (for example, the shape of a 7 versus a 1)
is lost during the evolution of the system. The neural network then receives a signal that can be described as
\textit{structured noise}, rather than semantically meaningful representations.

The fact that accuracy does not fall all the way to random levels indicates that some global properties are still preserved,
such as pixel density, approximate symmetries, or the number of connected components. However,
this information is insufficient for accurate classification.

We conclude that Conway’s Game of Life is not a suitable candidate as an \textit{embedding} function
for semantic classification tasks such as MNIST. This conclusion does not invalidate the experiment;
on the contrary, it constitutes a relevant empirical result.

\section{GoL as an Unsupervised clustering}
  


\bibliographystyle{icml2025}
\bibliography{report}


\end{document}
